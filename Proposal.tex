\documentclass[12pt]{article}
%Gummi|061|=)
% title page
\begin{document}
\title{Visualizing and Refining Connectivity Map Query Results}
\maketitle
\begin{titlepage}
\begin{center}
Proposal for a\\
Thesis in the Field of\\
Biotechnology\\[\baselineskip]
In Partial Fulfillment of the Requirements\\
for a Master of Liberal Arts Degree\\[\baselineskip]
Harvard University\\
Extension School\\[\baselineskip]
Theodore Natoli\\
51 Lewis Avenue, Apartment 3\\
Arlington, MA 02472\\
857-498-1946\\
\texttt{tnatoli@fas.harvard.edu}\\[\baselineskip]
Proposed Start Date: \today\\
Anticipated Date of Graduation: \today\\
Thesis Director: Aravind Subramanian
\end{center}
\end{titlepage}
% end of title stuff

\begin{abstract}
The Connectivity Map (CMap) is a database of gene expression signatures obtained from experiments in which cultured human cells are treated with pharmacologic and genomic perturbagens. A typical use case of this database is for a researcher to query with a signature of a cell state of interest and use the matching perturbagens to develop a functional hypothesis for follow-up. 

Current pattern matching algorithms that perform CMap queries suffer from a universal weakness -- the enormous size and richness of signatures in CMap means that a query typically generates hundreds of strong correlations which are hard to distinguish thereby making prioritization on the basis of a distance metric difficult.

We hypothesize that one mode of prioritization is to highlight query results that are highly interconnected amongst themselves over singletons. The goal of this work is to provide a web-based tool for visualizing CMap query results in a graph or network layout and to refine query results into more actionable hit lists.
\end{abstract}


\end{document}